\documentclass[a4paper]{article}
\usepackage{xcolor}
\usepackage[margin=1in]{geometry}
\usepackage{amsmath,amssymb}
\usepackage{lmodern,longtable,booktabs,array,url,xurl,hyperref,parskip}
\usepackage{lastpage}
\usepackage{fancyhdr}
\usepackage{siunitx}
\setcounter{secnumdepth}{1}

\fancypagestyle{plain}{
	\fancyhf{}
	\fancyhead[L]{B.B.F.M. Verspaandonk, M.Y.A. Wierckx}
	\fancyhead[R]{Cansat GAIA}
	\fancyfoot[C]{GROUND Lite Communications Protocol\hfill Page \thepage\ of \pageref{LastPage}}
}
\pagestyle{plain}
\renewcommand{\footrulewidth}{0.4pt}

%\hypersetup{hidelinks}
\hypersetup{colorlinks=true, linkcolor=blue, urlcolor=blue}

\title{GROUND Lite Communications Protocol Specification\\ \large Version 1.0.1}
\date{\today}
\author{B.B.F.M. Verspaandonk, M.Y.A. Wierckx}

\begin{document}

\maketitle

\tableofcontents

\section{Overview}
The GROUND (GAIA Radio OUtput Network Delivery) Lite protocol is a power and data efficiency optimized version of the original GROUND protocol. All data is serialized in little-endian format, meaning the least significant byte is sent first. For example, the number \texttt{0x1234} would be transmitted as \texttt{0x34 0x12}.

\section{Packet Structure}
A packet comprises the following fields:
\begin{longtable}{@{}lll@{}}
\toprule
\# & Field Name             & Size                         \\
\midrule
1  & \texttt{magic\_number} & 4 bytes                      \\
2  & \texttt{content\_type} & 1 bytes                      \\
3  & \texttt{content\_size} & 1 bytes                      \\
4  & \texttt{content}       & \texttt{content\_size} bytes \\
\bottomrule
\end{longtable}

\subsection{Field Descriptions}
\textbf{\texttt{magic\_number}}: A constant value \texttt{0x67616961} (ASCII for \texttt{gaia}) that marks the start of a packet.

\textbf{\texttt{content\_type}}: Specifies the type and structure of the data. Its interpreted as follows:
\begin{longtable}{@{}lrll@{}}
  \toprule
  Value         & Data Type         & Type                           & Description                      \\
  \midrule
  \texttt{0x01} & \texttt{float}    &\texttt{GPS\_POS[3]}            & GPS coordinates                          \\
  \texttt{0x02} & \texttt{float}    &\texttt{G\_FORCES[3]}           & G-force measurement                      \\
  \texttt{0x03} & \texttt{float}    &\texttt{ROTATION[3]}            & Angle measurement                        \\
  \texttt{0x04} & \texttt{uint32\_t}&\texttt{TIME}                   & Time                                     \\
  \texttt{0x05} & \texttt{uint32\_t}&\texttt{GPS\_FIX\_AGE}          & Time in ms since last gps fix            \\
  \texttt{0x06} & \texttt{float}    &\texttt{GPS\_HDOP}              & Horizontal Dilution of Precision         \\
  \texttt{0x07} & \texttt{uint8\_t} &\texttt{GPS\_NUM\_OF\_SATS}     & Number of satellites in view             \\
  \texttt{0x08} & \texttt{float}    &\texttt{GPS\_FAIL\_PERCENTAGE}  & Percentage of GPS cheksums failed        \\
  \texttt{0x09} & \texttt{uint16\_t}&\texttt{CO2\_CONCENTRATION}     & \emph{UNUSED} CO$_2$ concentration in ppb\\
  \texttt{0x0A} & \texttt{float}    &\texttt{TEMPERATURE}            & Temperature in $^\circ$C                 \\
  \texttt{0x0B} & \texttt{float    }&\texttt{PRESSURE}               & Pressure in Pa                           \\
  \texttt{0x0C} & \texttt{uint16\_t}&\texttt{DUST\_CONCENTRATION}    & Dust concentration in $\mu g/m^3$        \\
  \texttt{0x0D} & \texttt{float}    &\texttt{UV\_RADIATON}           & UV radiation in $mW/cm^2$                \\
  \texttt{0x0E} & \texttt{uint16\_t}&\texttt{PACKET\_NUM}             & Packet number                            \\
  \bottomrule
\end{longtable}

\textbf{\texttt{content\_size}}: The number of bytes in the \texttt{content} field.

\textbf{\texttt{content}}: The actual data payload. Its interpretation depends on \texttt{content\_type}.

\section{Handling \texttt{magic\_number} in Content}
If the \texttt{magic\_number} sequence \texttt{0x67616961} appears in the \texttt{content}, it must be escaped by appending a \texttt{0x00} byte immediately after. For example:

\texttt{67 61 69 61 }\textrightarrow\texttt{ 67 61 69 61 00}

The escape byte contributes to \texttt{content\_size} but should be removed during packet parsing.

\section{Encoding order}
The fields in a packet are encoded in the following order:
\begin{enumerate}
  \item Check if the \texttt{magic\_number} sequence appears in the \texttt{content} field. If so, escape it.
  \item Calculate the content size.
  \item Add the \texttt{magic\_number} sequence, \texttt{content\_type}, \texttt{content\_size} and \texttt{content} fields.
  \item Transmit the packet.
\end{enumerate}

\section{Examples}
\subsection{Single Value}
Packet encoding a single unsigned 16-bit integer with value \texttt{4660}:
\begin{verbatim}
  67 61 69 61 0B 04 00 50 7D 44
\end{verbatim}
\textbf{Breakdown:}
\begin{verbatim}
  67 61 69 61 // Magic number
  0B          // Content type: Pressure
  04          // Content size: 4 bytes
  00 50 7D 44 // Content: 1013.25 Pa
\end{verbatim}

\subsection{Array}
Packet encoding GPS coordinates (latitude, longitude) as two 64-bit doubles:
\begin{verbatim}
  67 61 69 61 03 0C 00 00 B4 42 9A 99 16 43 C3 F5 48 40
\end{verbatim}
\textbf{Breakdown:}
\begin{verbatim}
  67 61 69 61             // Magic number
  03                      // Content type: Angle measurement (float ROTATION[3])
  0C                      // Content size: 12 bytes
  00 00 B4 42             // Angle 1: 90.0
  9A 99 16 43             // Angle 2: 150.6
  C3 F5 48 40             // Angle 3: 3.14
\end{verbatim}

\subsection{Escaped Magic Number}
GPS coordinates with a \texttt{magic\_number} sequence in the content:
\begin{verbatim}
  67 61 69 61 01 0E 67 61 69 61 00 00 00 20 40 67 61 69 61 00
\end{verbatim}
\textbf{Breakdown:}
\begin{verbatim}
  67 61 69 61             // Magic Number
  01                      // Content type: GPS coordinates (float GPS_POS[3])
  0E                      // Content size: 14 bytes
  67 61 69 61 00          // Latitude with escaped magic number: 269069370000000000000 
                             (this isn't actually possible with GPS coordinates
                             but you get the point it's for demonstration)
  00 00 20 40             // Longitude: 2.5
  67 61 69 61 00          // Altitude with escaped magic number: 269069370000000000000
                             (about 28441 light years above sea level)
\end{verbatim}

\end{document}
