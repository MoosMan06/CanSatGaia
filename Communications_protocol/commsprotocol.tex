\documentclass[a4paper]{article}
\usepackage{xcolor}
\usepackage[margin=1in]{geometry}
\usepackage{amsmath,amssymb}
\usepackage{lmodern,longtable,booktabs,array,url,xurl,hyperref, parskip}
%\hypersetup{hidelinks}
\hypersetup{colorlinks=true, linkcolor=blue, urlcolor=blue}

\title{GROUND Communications Protocol Specification\\ \large Version 1.1}
\date{\today}
\author{B.B.F.M. Verspaandonk}

\begin{document}

\maketitle

\section*{Overview}
The GROUND (GAIA Radio OUtput Network Delivery) protocol is designed for transmitting data from the Cansat to the ground station. All data is serialized in little-endian format, meaning the least significant byte is sent first. For example, the number \texttt{0x1234} would be transmitted as \texttt{0x34 0x12}.

\section*{Packet Structure}
A packet comprises the following fields:
\begin{longtable}{@{}lll@{}}
\toprule
\# & Field Name         & Size                \\
\midrule
1  & \texttt{magic}      & 4 bytes             \\
2  & \texttt{content\_type} & 2 bytes             \\
3  & \texttt{content\_size} & 2 bytes             \\
4  & \texttt{content}     & \texttt{content\_size} bytes \\
\bottomrule
\end{longtable}

\subsection*{Field Descriptions}
\textbf{\texttt{magic}}: A constant value \texttt{0x47414941} (ASCII for \texttt{GAIA}) that marks the start of a packet.

\textbf{\texttt{content\_type}}: Specifies the type and structure of the data. Its two bytes are interpreted as follows:
\begin{itemize}
  \item \textbf{First byte}:
  \begin{itemize}
    \item High nibble (\texttt{0xF0}): Indicates if the content includes a CRC checksum (\texttt{0x0} = no, \texttt{0x1} = yes).
    \item Low nibble (\texttt{0x0F}): Specifies the data category  (see \hyperref[data-types]{Data Types}). May never be \texttt{0x0}.
  \end{itemize}
  \item \textbf{Second byte}:
  \begin{itemize}
    \item High nibble (\texttt{0xF0}): Indicates whether the data is a single value (\texttt{0x0}) or an array (\texttt{0x1}).
    \item Low nibble (\texttt{0x0F}): Specifies the data's primitive type (see \hyperref[data-types]{Data Types}).
  \end{itemize}
\end{itemize}

\textbf{\texttt{content\_size}}: The number of bytes in the \texttt{content} field.

\textbf{\texttt{content}}: The actual data payload. Its interpretation depends on \texttt{content\_type}.

\section*{Handling \texttt{magic} in Content}
If the \texttt{magic} sequence \texttt{0x47414941} appears in the \texttt{content}, it must be escaped by appending a \texttt{0x00} byte immediately after. For example:

\texttt{47 41 49 41 }\textrightarrow\texttt{ 47 41 49 41 00}

The escape byte contributes to \texttt{content\_size} but is excluded from array length calculations. It should be removed during packet parsing.

\section*{Data Types}\label{data-types}
\subsection*{Raw Data Types}
\begin{longtable}{@{}lll@{}}
\toprule
Value    & Type             & Description                  \\
\midrule
\texttt{0x00} & \texttt{u8}       & Unsigned 8-bit integer       \\
\texttt{0x01} & \texttt{u16}      & Unsigned 16-bit integer      \\
\texttt{0x02} & \texttt{u32}      & Unsigned 32-bit integer      \\
\texttt{0x03} & \texttt{u64}      & Unsigned 64-bit integer      \\
\texttt{0x04} & \texttt{s8}       & Signed 8-bit integer         \\
\texttt{0x05} & \texttt{s16}      & Signed 16-bit integer        \\
\texttt{0x06} & \texttt{s32}      & Signed 32-bit integer        \\
\texttt{0x07} & \texttt{s64}      & Signed 64-bit integer        \\
\texttt{0x08} & \texttt{float}    & 32-bit floating point number \\
\texttt{0x09} & \texttt{double}   & 64-bit floating point number \\
\texttt{0x0A} & \texttt{bool}     & Boolean                      \\
\texttt{0x0B} & \texttt{char}     & ASCII character              \\
\bottomrule
\end{longtable}

\subsection*{Categorical Data Types}
\begin{longtable}{@{}lll@{}}
\toprule
Value    & Type             & Description                  \\
\midrule
\texttt{0x01} & \texttt{GPS}        & GPS coordinates              \\
\texttt{0x02} & \texttt{G-force}    & G-force measurement          \\
\texttt{0x03} & \texttt{Angle}      & Angle measurement            \\
\texttt{0x04} & \texttt{Time}       & GPS Time                     \\
\texttt{0x05} & \texttt{Age}        & Time in ms since last gps fix\\
\bottomrule
\end{longtable}

\section*{Examples}
\subsection*{Single Value}
Packet encoding a single unsigned 16-bit integer with value \texttt{4660}:
\begin{verbatim}
47 41 49 41 21 01 02 00 34 12
\end{verbatim}
\textbf{Breakdown:}
\begin{verbatim}
47 41 49 41 // Magic number
21          // Content type: Unsigned 16-bit integer
02 00       // Content size: 2 bytes
34 12       // Content: 4660
\end{verbatim}

\subsection*{Array}
Packet encoding GPS coordinates (latitude, longitude) as two 64-bit doubles:
\begin{verbatim}
  47 41 49 41 11 19 11 00 8C 01 E2 36 9D B4 49 40 1B F1 90 7B 96 F4 15 40 6D
\end{verbatim}
\textbf{Breakdown:}
\begin{verbatim}
  47 41 49 41             // Magic number
  11                      // Content category: GPS coordinates with CRC
  19                      // Content type: Array of 64-bit doubles
  10 00                   // Content size: 17 bytes
  1F 8B 8C 0C F4 B6 49 40 // Latitude: 51.411047802309525
  77 05 98 4A 64 D4 15 40 // Longitude: 5.488855295869395
  6D                      // CRC-8 checksum
\end{verbatim}

\subsection*{Escaped Magic Number}
GPS coordinates with a \texttt{magic} sequence in the content:
\begin{verbatim}
  47 41 49 41 11 19 12 00 47 41 49 41 00 F8 B6 49 40 10 61 4A 8F 35 D4 15 40 6D
\end{verbatim}
\textbf{Breakdown:}
\begin{verbatim}
  47 41 49 41                // Magic number
  11                         // Content category: GPS coordinates with CRC
  19                         // Content type: Array of 64-bit doubles
  12 00                      // Content size: 17 bytes
  47 41 49 41 00 F8 B6 49 40 // Latitude: 51.429451142090834 (with escaped magic number)
  10 61 4A 8F 35 D4 15 40    // Longitude: 5.457235564150565
  6D                         // CRC-8 checksum
\end{verbatim}

\end{document}
